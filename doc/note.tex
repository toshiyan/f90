%////////////////////////////////////////////////////////////////////////////////////////////////////%
% PREAMBLES 
%////////////////////////////////////////////////////////////////////////////////////////////////////%

% Style %
\documentclass{article}

% Packages %
\usepackage{amsmath,amssymb,bm,color,fancyhdr,graphicx,Macro,mathrsfs,tabularx}

\def\grad{\phi}
\def\curl{\varpi}
\def\estx{\widehat{x}}

\def\T{\Theta}
\def\tT{\widetilde{\T}}
\def\tE{\widetilde{E}}
\def\tB{\widetilde{B}}
\def\hX{\widehat{X}}
\def\hY{\widehat{Y}}
\def\hCTT{\widehat{C}^{\rm \T\T}}
\def\hCTE{\widehat{C}^{\rm \T E}}
\def\hCEE{\widehat{C}^{\rm EE}}
\def\hCEB{\widehat{C}^{\rm EB}}
\def\hCBB{\widehat{C}^{\rm BB}}
\def\hCXX{\widehat{C}^{\rm XX}}
\def\hCXY{\widehat{C}^{\rm XY}}
\def\hCYY{\widehat{C}^{\rm YY}}
\def\CTT{C^{\rm \T\T}}
\def\CTE{C^{\rm \T E}}
\def\CEE{C^{\rm EE}}
\def\CBB{C^{\rm BB}}
\def\W{W}

\def\z{\zeta}
\def\be{\bm{e}}
\def\intn{\Int{2}{\hatn}{}}
\def\sn{\mathbin{\text{\rotatebox[origin=c]{180}{$\bn$}}}}
\def\conv{\widetilde{\sum_{LM\l'm'}}^{(\l m)}}

% page style
\renewcommand{\rmdefault}{ptm}
\setlength{\topmargin}{-0.8cm}
\setlength{\oddsidemargin}{-0.5cm}
\setlength{\evensidemargin}{0.5cm}
\setlength{\textheight}{23.0cm}
\setlength{\textwidth}{16.0cm}

% Header and Footer %
\pagestyle{fancy}
\lhead{\leftmark}
\rhead{\rightmark}
\cfoot{\thepage}
\setlength{\headheight}{10pt}
\renewcommand{\sectionmark}[1]{\markright{}{}}

%////////////////////////////////////////////////////////////////////////////////////////////////////%
% DOCUMENTS 
%////////////////////////////////////////////////////////////////////////////////////////////////////%

\begin{document}

% Title %
\title{Fast Computation for the Quadratic Estimator Normalization}

% Authors %
\author{Toshiya Namikawa}

% Date %
\date{{\it Latest revision} : \today}

% Abstract %

\maketitle

% Table of contents %
\tableofcontents

% Contents %

%////////////////////////////////////////////////////////////////////////////////////////////////////%
\section{Preliminaries} 
%////////////////////////////////////////////////////////////////////////////////////////////////////%

Here I will generalize the fast modal algorithm of \cite{Dvorkin:2009ah,Smith:2010gu} 
to the case of polarization angle rotation, patchy reionization and others. 
In the followings, for multipoles of the CMB anisotropies, we use small letters (e.g., $\l$), 
while large letters are used for multipoles of the distortion fields (lensing, rotation, etc). 

%::::::::::::::::::::::::::::::::::::::::::::::::::::::::::::::::::::::::::::::::::::::::::::::::::::%
\subsection{CMB} 
%::::::::::::::::::::::::::::::::::::::::::::::::::::::::::::::::::::::::::::::::::::::::::::::::::::%

The CMB temperature fluctuations are denoted as $\T$ and the CMB linear polarization is expressed by 
the Stokes parameters, $Q$ and $U$. The harmonic coefficients of the temperature anisotropies 
(and, in general, any scalar quantities $x$) are given by 
%----------------------------------------------------------------------------------------------------% 
\al{
	x_{LM} = \intn Y_{LM}^*(\hatn) x (\hatn)  \,. \label{Eq:scalalm}
}
%----------------------------------------------------------------------------------------------------% 
where $Y_{LM}$ is the spin-0 spherical harmonics. 
On the other hand, the values of $Q$ and $U$ are changed by the rotation of the sphere. 
These Stokes parameters are therefore usually transformed into the rotational invariant quantities, 
the $E$ and $B$ modes, as
%----------------------------------------------------------------------------------------------------%
\al{
	[E \pm \iu B ]_{\l m} = \intn (Y_{\l m}^{\pm 2})^*(\hatn) [Q\pm \iu U](\hatn)  \,. 
}
%----------------------------------------------------------------------------------------------------% 
Here, $Y_{\l m}^{\pm 2}$ is the spin-2 spherical harmonics. For short notation, we also use 
%----------------------------------------------------------------------------------------------------% 
\al{
	\Xi^\pm &= E \pm\iu B \,, \notag \\ P^\pm &= Q\pm\iu U
}
%----------------------------------------------------------------------------------------------------% 

%::::::::::::::::::::::::::::::::::::::::::::::::::::::::::::::::::::::::::::::::::::::::::::::::::::%
\subsection{Lensing} 
%::::::::::::::::::::::::::::::::::::::::::::::::::::::::::::::::::::::::::::::::::::::::::::::::::::%

The lensing effect on CMB anisotropies is described as remapping of the unlensed CMB anisotropies 
by the deflection angle, 
%----------------------------------------------------------------------------------------------------%
\al{
	X(\hatn) = X(\hatn+\bm{d}) \,,
}
%----------------------------------------------------------------------------------------------------%
where $X$ is $\T$ or $P^\pm$. 

The deflection angle of the CMB lensing is decomposed into the lensing potential, $\grad$, and 
curl mode, $\curl$, as
%----------------------------------------------------------------------------------------------------% 
\al{
	\bm{d} = \bn \grad + \sn \curl \,,
}
%----------------------------------------------------------------------------------------------------% 
where the operator $\sn=\star\bn$ denotes the derivatives with $90^\circ$ rotation counterclockwise on 
the plane perpendicular to the line-of-sight direction and then operation. 
The harmonic coefficients of $\grad$ and $\curl$ are given by \eq{Eq:scalalm}. 
The remapping of the CMB anisotropies is then given by
%----------------------------------------------------------------------------------------------------%
\al{
	X(\hatn) = X(\hatn) + [\bn\grad + \sn\curl]\cdot\bn X + \mC{O}(\grad^2,\curl^2)
	\,.
}
%----------------------------------------------------------------------------------------------------%


%::::::::::::::::::::::::::::::::::::::::::::::::::::::::::::::::::::::::::::::::::::::::::::::::::::%
\subsection{Rotation} 
%::::::::::::::::::::::::::::::::::::::::::::::::::::::::::::::::::::::::::::::::::::::::::::::::::::%

If the rotation angle is small, the modulation of polarization after rotation by an angle $\alpha$ 
is given by (e.g. \cite{Gluscevic:2009})
%----------------------------------------------------------------------------------------------------%
\al{
	\delta P^\pm = \pm 2\alpha P^\pm \,.
}
%----------------------------------------------------------------------------------------------------%
The harmonic coefficients of $\alpha$ is given by \eq{Eq:scalalm}. 


%::::::::::::::::::::::::::::::::::::::::::::::::::::::::::::::::::::::::::::::::::::::::::::::::::::%
\subsection{Inhomogeneous Reionization} 
%::::::::::::::::::::::::::::::::::::::::::::::::::::::::::::::::::::::::::::::::::::::::::::::::::::%

The inhomogeneities of the reionization could vary the optical depth $\tau$ across the CMB sky. 
If the spatial variation of $\tau$ is very small, this leads to the modulation in CMB temperature 
and polarization as (e.g. \cite{Dvorkin:2009ah,Gluscevic:2012qv})
%----------------------------------------------------------------------------------------------------%
\al{
	\T \to \T + \tau \T \,.
	P^\pm \to P^\pm + \tau P^\pm \,.
}
%----------------------------------------------------------------------------------------------------%
The harmonic coefficients of $\tau$ is given by \eq{Eq:scalalm}. 


%::::::::::::::::::::::::::::::::::::::::::::::::::::::::::::::::::::::::::::::::::::::::::::::::::::%
\subsection{Spherical Harmonics and Wigner-3j} 
%::::::::::::::::::::::::::::::::::::::::::::::::::::::::::::::::::::::::::::::::::::::::::::::::::::%

The spherical harmonics is related to the Wigner-3j symbols as \cite{QTAM}
%----------------------------------------------------------------------------------------------------%
\al{
	\intn Y^{s_1}_{\l_1m_1}Y^{s_2}_{\l_2m_2}Y^{s_3}_{\l_3m_3} 
		= \sqrt{\frac{(2\l_1+1)(2\l_2+1)(2\l_3+1)}{4\pi}} 
		\Wjm{\l_1}{\l_2}{\l_3}{-s_1}{-s_2}{-s_3} \Wjm{\l_1}{\l_2}{\l_3}{m_1}{m_2}{m_3} 
	\,. \label{3times}
}
%----------------------------------------------------------------------------------------------------%

%::::::::::::::::::::::::::::::::::::::::::::::::::::::::::::::::::::::::::::::::::::::::::::::::::::%
\subsection{Derivatives of Spherical Harmonics} 
%::::::::::::::::::::::::::::::::::::::::::::::::::::::::::::::::::::::::::::::::::::::::::::::::::::%

In general, denoting $a_\l^s=\sqrt{(\l-s)(\l+s)/2}$, the derivative of the spherical 
harmonics is given by
%----------------------------------------------------------------------------------------------------%
\al{
	\bn Y^s_{\l m} = a^s_\l Y^{s+1}_{\l m} \be^* - a^{-s}_\l Y^{s-1}_{\l m} \be  \,. 
}
%----------------------------------------------------------------------------------------------------%
Here, we introduce the polarization vector $\be$ which are defined 
%----------------------------------------------------------------------------------------------------%
\al{
	\be = \frac{\be_1+\iu\be_2}{\sqrt{2}}
}
%----------------------------------------------------------------------------------------------------%
with $\be_i$ denoting the basis vectors orthogonal to the radial vector. The polarization vector satisfies 
$\be\cdot\be=0$, $\be\cdot\be^*=1$, $\star\be=-\iu\be$. In particular, for $s=0$, 
%----------------------------------------------------------------------------------------------------%
\al{
	\bn Y_{\l m} = a^0_\l \big( Y^1_{\l m} \be^* - Y^{-1}_{\l m} \be \big)  \,, 
}
%----------------------------------------------------------------------------------------------------%
and, for $s=\pm 2$, denoting $a^\pm=a_\l^{\pm 2}$, 
%----------------------------------------------------------------------------------------------------%
\al{
	\bn Y^2_{\l m}    &= a^+_\l Y^3_{\l m} \be^* - a^-_\l Y^1_{\l m} \be   \,, \notag \\ 
	\bn Y^{-2}_{\l m} &= a^-_\l Y^{-1}_{\l m} \be^* - a^+_\l Y^{-3}_{\l m} \be   \,. 
}
%----------------------------------------------------------------------------------------------------%

%::::::::::::::::::::::::::::::::::::::::::::::::::::::::::::::::::::::::::::::::::::::::::::::::::::%
\subsection{Map derivatives} 
%::::::::::::::::::::::::::::::::::::::::::::::::::::::::::::::::::::::::::::::::::::::::::::::::::::%

Derivative of scalar quantities such as the CMB temperature fluctuations and lensing potential is  
%----------------------------------------------------------------------------------------------------%
\al{
	\bn x &= \sum_{LM} x_{LM}\bn Y_{LM} 
		= \sum_{LM} x_{LM} a^0_L \left( Y^1_{LM}\be^* - Y^{-1}_{LM}\be \right) 
		= x^+ \be^* - x^- \be
	\,. 
}
%----------------------------------------------------------------------------------------------------%
where we define
%----------------------------------------------------------------------------------------------------%
\al{
	x^\pm &\equiv \sum_{LM} x_{LM} a^0_L Y^{\pm 1}_{LM} \,,
}
%----------------------------------------------------------------------------------------------------%
and $(x^+)^*=-x^-$. The rotation of a pseudo-scalar quantity is given by
%----------------------------------------------------------------------------------------------------%
\al{
	\sn \curl &= \sum_{LM} \curl_{LM} \sn Y_{LM} 
		= \sum_{LM} \curl_{LM} a^0_L \iu \left( Y^1_{LM} \be^* + Y^{-1}_{LM}\be \right) 
		= \iu (\curl^+ \be^* + \curl^- \be)
	\,, 
}
%----------------------------------------------------------------------------------------------------%
and $(\curl^+)^*=-\curl^-$. Spin-2 fields such as the CMB linear polarization is given by 
%----------------------------------------------------------------------------------------------------%
\al{
	\bn P^+ &= \sum_{\l m} \Xi^+_{\l m}\bn Y^2_{\l m} 
		= \sum_{\l m} \Xi^+_{\l m} \left( a^+_\l Y^3_{\l m}\be^* - a^-_\l Y^1_{\l m}\be\right) 
		= \Xi^{+^+} \be^* - \Xi^{+^-} \be
	\,, \\ 
	\bn P^- &= (\bn P^+)^* = \sum_{\l m} \Xi^-_{\l m}\bn Y^{-2}_{\l m}
		= \sum_{\l m} \Xi^-_{\l m} \left( a^-_\l Y^{-1}_{\l m}\be^* - a^+_\l Y^{-3}_{\l m}\be\right)
		= \Xi^{-^+} \be^* - \Xi^{-^-} \be 
	\,. 
}
%----------------------------------------------------------------------------------------------------%
Note that $(\Xi^{+^+})^*=-\Xi^{-^-}$ and $(\Xi^{+^-})^*=-\Xi^{-^+}$. 

\newpage


%////////////////////////////////////////////////////////////////////////////////////////////////////%
\section{Distortion of CMB anisotropies} 
%////////////////////////////////////////////////////////////////////////////////////////////////////%

In the following, we first define useful quantities to compute the distortion effect. 
The parity symmetry indicator is given by
%----------------------------------------------------------------------------------------------------%
\al{
	p^\pm_{\l_1\l_2\l_3}  &\equiv \frac{1\pm (-1)^{\l_1+\l_2+\l_3}}{2}  \,, \\
}
%----------------------------------------------------------------------------------------------------%
An even (odd) parity quantity contains $p^+$ ($p^-$). 
A multipole factor is defined as 
%----------------------------------------------------------------------------------------------------%
\al{
	\gamma_{\l_1\l_2\l_3} &\equiv \sqrt{\frac{(2\l_1+1)(2\l_2+1)(2\l_3+1)}{4\pi}}  \,. 
}
%----------------------------------------------------------------------------------------------------%
The convolutuon operator in full sky is defined as
%----------------------------------------------------------------------------------------------------%
\al{
	\conv \equiv \sum_{LM\l'm'}(-1)^m\Wjm{\l}{L}{\l'}{-m}{M}{m'} \,.
}
%----------------------------------------------------------------------------------------------------%


%::::::::::::::::::::::::::::::::::::::::::::::::::::::::::::::::::::::::::::::::::::::::::::::::::::%
\subsection{Lensing distortion} 
%::::::::::::::::::::::::::::::::::::::::::::::::::::::::::::::::::::::::::::::::::::::::::::::::::::%

The lensing contributions in the position space become 
%----------------------------------------------------------------------------------------------------%
\al{
	\delta^\grad \T &= \bn\grad\cdot\bn\T = - \grad^-\T^+ - \grad^+\T^- 
	\,, \notag \\ 
	\delta^\curl \T &= \sn\curl\cdot\bn\T = \iu (\curl^-\T^+ - \curl^+\T^-) 
	\,, \notag \\ 
	\delta^\grad P^\pm &= \bn\grad\cdot\bn P^\pm = - \grad^-\Xi^{\pm ^+} - \grad^+\Xi^{\pm ^-} 
	\,, \notag \\
	\delta^\curl P^\pm &= \sn\curl\cdot\bn P^\pm = \iu (\curl^-\Xi^{\pm ^+} - \curl^+\Xi^{\pm ^-}) 
	\,. 
}
%----------------------------------------------------------------------------------------------------%

%++++++++++++++++++++++++++++++++++++++++++++++++++++++++++++++++++++++++++++++++++++++++++++++++++++%
\subsubsection{Lens distortion in harmonic space: Temperature} 
%++++++++++++++++++++++++++++++++++++++++++++++++++++++++++++++++++++++++++++++++++++++++++++++++++++%

The harmonics transform of the lensing contributions is 
%----------------------------------------------------------------------------------------------------%
\al{
	\delta^\grad \T_{\l m} &= -\intn Y^*_{\l m}[\grad^-\T^+ + \grad^+\T^-] 
	\notag \\ 
		&= -\sum_{LM \l' m'}\grad_{LM}\T_{\l' m'}a_L^0 a_{\l'}^0 
		\intn (-1)^m Y_{\l,-m}[Y^{-1}_{LM}Y^1_{\l' m'}+Y^1_{LM}Y^{-1}_{\l' m'}]
	\notag \\ 
		&= -\sum_{LM\l'm'} \grad_{LM}\T_{\l'm'} 2a_L^0 a_{\l'}^0 
			p^+_{\l L\l'}\gamma_{\l L\l'} (-1)^m\Wjm{\l}{L}{\l'}{-m}{M}{m'}\Wjm{\l}{L}{\l'}{0}{1}{-1}
	\notag \\ 
		&= -\conv \grad_{LM}\T_{\l'm'} 2a_L^0 a_{\l'}^0 
			p^+_{\l L\l'}\gamma_{\l L\l'}\Wjm{\l}{L}{\l'}{0}{1}{-1}
	\notag \\ 
		&= \conv \grad_{LM} \T_{\l'm'} \W^{\grad,0}_{\l L\l'}
	\,. 
}
%----------------------------------------------------------------------------------------------------%
Here we introduce coefficients $c_\grad=1$ and $c_\curl=-\iu$, and denote
%----------------------------------------------------------------------------------------------------%
\al{
	\W_{\l_1\l_2\l_3}^{\grad,0} = - 2 c_\grad a_{\l_2}^0 a_{\l_3}^0
		p^+_{\l_1\l_2\l_3}\gamma_{\l_1\l_2\l_3}\Wjm{\l_1}{\l_2}{\l_3}{0}{1}{-1} 
	\,.
}
%----------------------------------------------------------------------------------------------------%
Note that $(\W_{\l_1\l_2\l_3}^{\grad,0})^*=\W_{\l_1\l_2\l_3}^{\grad,0}$. 

On the other hand, for curl mode, 
%----------------------------------------------------------------------------------------------------%
\al{
	\delta^\curl \T_{\l m} 
	&= \iu \intn Y^*_{\l m}[\curl^-\T^+ - \curl^+\T^-] 
		\notag \\ 
	&= \sum_{LM\l'm'} \curl_{LM}\T_{\l'm'} 2\iu a_L^0 a_{\l'}^0
		p^-_{\l L\l'} \gamma_{\l L\l'} (-1)^m \Wjm{\l}{L}{\l'}{-m}{M}{m'} \Wjm{\l}{L}{\l'}{0}{1}{-1} 
		\notag \\ 
	&= \conv \curl_{LM}\T_{\l'm'} \W^{\curl,0}_{\l L\l'}
	\,, 
}
%----------------------------------------------------------------------------------------------------%
with 
%----------------------------------------------------------------------------------------------------%
\al{
	\W_{\l_1\l_2\l_3}^{\curl,0} = -2c_\curl a_{\l_2}^0 a_{\l_3}^0 
		p^-_{\l_1\l_2\l_3} \gamma_{\l_1\l_2\l_3} \Wjm{\l_1}{\l_2}{\l_3}{0}{1}{-1} 
	\,. \label{Eq:def-S0c}
}
%----------------------------------------------------------------------------------------------------%
Note that the above quantity is consistent with Ref.~\cite{Namikawa:2011b} and also 
$(\W_{\l_1\l_2\l_3}^{\curl,0})^*=(-1)^{\l_1+\l_2+\l_3}\W_{\l_1\l_2\l_3}^{\curl,0}$ which is consistent 
with \eqref{Eq:def-S0c}. 


%++++++++++++++++++++++++++++++++++++++++++++++++++++++++++++++++++++++++++++++++++++++++++++++++++++%
\subsubsection{Lens distortion in harmonic space: Polarization} 
%++++++++++++++++++++++++++++++++++++++++++++++++++++++++++++++++++++++++++++++++++++++++++++++++++++%

The lensed anisotropies for polarizations are given by 
%----------------------------------------------------------------------------------------------------%
\al{
	\delta^\grad \Xi^\pm_{\l m} 
	&= - \intn (Y^{\pm 2}_{\l m})^* [\grad^- \Xi^{\pm^+}+\grad^+ \Xi^{\pm^-}] 
		\notag \\ 
	&= - \sum_{LM\l'm'} \grad_{LM}\Xi^\pm_{\l'm'}a_L^0 \intn (Y^{\pm 2}_{\l m})^* 
		[a_{\l'}^+ Y^{\mp 1}_{LM}Y^{\pm 3}_{\l'm'} + a_{\l'}^- Y^{\pm 1}_{LM}Y^{\pm 1}_{\l'm'}]
		\notag \\ 
	&= - \sum_{LM\l'm'}(-1)^m \Wjm{\l}{L}{\l'}{-m}{M}{m'} \grad_{LM}\Xi^\pm_{\l'm'}
		\gamma_{\l L\l'} a_L^0 \left[a_{\l'}^+ \Wjm{\l}{L}{\l'}{\mp 2}{\mp 1}{\pm 3} 
		+ a_{\l'}^- \Wjm{\l}{L}{\l'}{\mp 2}{\pm 1}{\pm 1}\right]
		\notag \\ 
	&= \conv \grad_{LM}\Xi^\pm_{\l'm'} \W^{\grad,\pm 2}_{\l L\l'}
	\,, 
}
%----------------------------------------------------------------------------------------------------%
with 
%----------------------------------------------------------------------------------------------------%
\al{
	S^{\grad,2}_{\l_1\l_2\l_3} &= (-1)^{\l_1+\l_2+\l_3} S^{\grad,-2}_{\l_1\l_2\l_3} 
		= - c_\grad \gamma_{\l_1\l_2\l_3} a_{\l_2}^0 
			\left[ a_{\l_3}^+ \Wjm{\l_1}{\l_2}{\l_3}{-2}{-1}{3} 
			+ a_{\l_3}^- \Wjm{\l_1}{\l_2}{\l_3}{-2}{1}{1} \right]
	\,. 
}
%----------------------------------------------------------------------------------------------------%
For curl mode, 
%----------------------------------------------------------------------------------------------------%
\al{
	\delta^\curl\Xi^\pm_{\l m} 
	&= \iu\intn (Y^{\pm 2}_{\l m})^* [\curl^-\Xi^{\pm^+}-\curl^+\Xi^{\pm^-}] 
		\notag \\ 
	&= \pm \iu\sum_{LM\l'm'} (-1)^m \Wjm{\l}{L}{\l'}{-m}{M}{m'} \curl_{LM}\Xi^{\pm}_{\l'm'}
		a_L^0 \gamma_{\l L\l'} \left[ a_{\l'}^+ \Wjm{\l}{L}{\l'}{\mp 2}{\mp 1}{\pm 3} 
		- a_{\l'}^- \Wjm{\l}{L}{\l'}{\mp 2}{\pm 1}{\pm 1} \right]
		\notag \\ 
	&= \conv \curl_{LM}\Xi^\pm_{\l'm'} \W^{\curl,\pm 2}_{\l L\l'}
	\,, 
}
%----------------------------------------------------------------------------------------------------%
with 
%----------------------------------------------------------------------------------------------------%
\al{
	\W^{\curl,2}_{\l_1\l_2\l_3} &= -(-1)^{\l_1+\l_2+\l_3} \W^{\curl,-2}_{\l_1\l_2\l_3} 
		= - c_\curl \gamma_{\l_1\l_2\l_3} a_{\l_2}^0 
			\left[a_{\l_3}^+ \Wjm{\l_1}{\l_2}{\l_3}{-2}{-1}{3} 
			- a_{\l_3}^- \Wjm{\l_1}{\l_2}{\l_3}{-2}{1}{1}\right] 
	\,. 
}
%----------------------------------------------------------------------------------------------------%

Now we consider the lensed E/B modes separately. In general, for $X^\pm=A\pm\iu B=(a\pm\iu b)c^{(\pm)}$, 
%----------------------------------------------------------------------------------------------------%
\al{
	A &= \frac{X^++X^-}{2} = \left(a\frac{c^{(+)}+c^{(-)}}{2}+\iu b\frac{c^{(+)}-c^{(-)}}{2}\right) 
	\,, \\ 
	B &= \frac{X^+-X^-}{2\iu} = \left(-a\iu\frac{c^{(+)}-c^{(-)}}{2}+b\frac{c^{(+)}+c^{(-)}}{2}\right) 
}
%----------------------------------------------------------------------------------------------------%
The lensing correction terms for E/B modes are then given by 
%----------------------------------------------------------------------------------------------------%
\al{
	\delta^x E_{\l m} 
		&= \conv \grad_{LM} \left[\W^{x,+}_{\l L\l'} E_{\l'm'} + \W^{x,-}_{\l L\l'} B_{\l'm'} \right]
	\,, \\ 
	\delta^x B_{\l m} 
		&= \conv \grad_{LM} \left[-\W^{x,-}_{\l L\l'} E_{\l'm'} + \W^{x,+}_{\l L\l'} B_{\l'm'} \right]
	\,. 
}
%----------------------------------------------------------------------------------------------------%
Here we define 
%----------------------------------------------------------------------------------------------------%
\al{
	\W^{x,+}_{\l_1\l_2\l_3} &\equiv \frac{\W^{x,2}_{\l_1\l_2\l_3} + \W^{x,-2}_{\l_1\l_2\l_3}}{2}
		= \frac{1+c_x^2(-1)^{\l_1+\l_2+\l_3}}{2} \W^{x,2}_{\l_1\l_2\l_3}
	\notag \\ 
		&= - \wp^{x,+}_{\l_1\l_2\l_3} \gamma_{\l_1\l_2\l_3} a_{\l_2}^0 
		\left[ a_{\l_3}^+ \Wjm{\l_1}{\l_2}{\l_3}{-2}{-1}{3} 
		+ c_x^2 a_{\l_3}^- \Wjm{\l_1}{\l_2}{\l_3}{-2}{1}{1} \right]
	\,, \notag \\ 
	\W^{x,-}_{\l_1\l_2\l_3} &\equiv \iu \frac{ \W^{x,2}_{\l_1\l_2\l_3} - \W^{x,-2}_{\l_1\l_2\l_3}}{2}
		= \iu \frac{1-c_x^2(-1)^{\l_1+\l_2+\l_3}}{2} \W^{x,2}_{\l_1\l_2\l_3}
	\notag \\ 
		&= - \wp^{x,-}_{\l_1\l_2\l_3} \gamma_{\l_1\l_2\l_3} a_{\l_2}^0 
		\left[ a_{\l_3}^+ \Wjm{\l_1}{\l_2}{\l_3}{-2}{-1}{3} 
		+ c_x^2 a_{\l_3}^- \Wjm{\l_1}{\l_2}{\l_3}{-2}{1}{1} \right]
	\,, 
}
%----------------------------------------------------------------------------------------------------%
where we reintroduce a parity indicator as 
%----------------------------------------------------------------------------------------------------%
\al{
	\wp^{x,+}_{\l_1\l_2\l_3} &= c_x \frac{1+c_x^2(-1)^{\l_1+\l_2+\l_3}}{2}      \,, \notag \\ 
	\wp^{x,-}_{\l_1\l_2\l_3} &= \iu c_x \frac{1-c_x^2(-1)^{\l_1+\l_2+\l_3}}{2}  \,. 
}
%----------------------------------------------------------------------------------------------------%


%::::::::::::::::::::::::::::::::::::::::::::::::::::::::::::::::::::::::::::::::::::::::::::::::::::%
\subsection{Rotation distortion} 
%::::::::::::::::::::::::::::::::::::::::::::::::::::::::::::::::::::::::::::::::::::::::::::::::::::%

The E and B modes after the rotation are given by
%----------------------------------------------------------------------------------------------------%
\al{
	\delta \Xi^\pm 
	&= \pm 2 \intn (Y^{\pm 2}_{\l m})^* \alpha P^\pm
		\notag \\ 
	&= \pm 2 \sum_{LM\l'm'} \alpha_{LM}\Xi^\pm_{\l'm'} \intn (Y^{\pm 2}_{\l m})^*Y_{LM}Y^{\pm 2}_{\l'm'}
		\notag \\ 
	&= \pm 2 \sum_{LM\l'm'} (-1)^m \Wjm{\l}{L}{\l'}{-m}{M}{m'} \alpha_{LM}\Xi^\pm_{\l'm'}
		\gamma_{\l L\l'} \Wjm{\l}{L}{\l'}{\mp 2}{0}{\pm 2} 
		\notag \\ 
	&= \conv \alpha_{LM} \Xi^\pm_{\l'm'} \W^{\alpha,\pm 2}_{\l L\l'}
	\,, 
}
%----------------------------------------------------------------------------------------------------%
with
%----------------------------------------------------------------------------------------------------%
\al{
	\W^{\alpha,\pm 2}_{\l_1\l_2\l_3} = \pm 2\gamma_{\l_1\l_2\l_3}\Wjm{\l_1}{\l_2}{\l_3}{\mp 2}{0}{\pm 2} 
	\,. 
}
%----------------------------------------------------------------------------------------------------%
The distorted E and B modes are then described as
%----------------------------------------------------------------------------------------------------%
\al{
	\delta E_{\l m} &= \conv \alpha_{LM} 
		\left(E_{\l'm'} \W^{\alpha,+}_{\l L\l'} + B_{\l'm'} \W^{\alpha,-}_{\l L\l'}\right) 
	\,, \\ 
	\delta B_{\l m} &= \conv \alpha_{LM}
		\left(-E_{\l'm'} \W^{\alpha,-}_{\l L\l'} + B_{\l'm'} \W^{\alpha,+}_{\l L\l'}\right) 
}
%----------------------------------------------------------------------------------------------------%
where we define $c_\alpha=1$ and
%----------------------------------------------------------------------------------------------------%
\al{
	\W^{\alpha,\pm}_{\l_1\l_2\l_3} 
		= 2 \wp^{\alpha,\mp}_{\l_1\l_2\l_3} \gamma_{\l_1\l_2\l_3} \Wjm{\l_1}{\l_2}{\l_3}{-2}{0}{2}
}
%----------------------------------------------------------------------------------------------------%

%::::::::::::::::::::::::::::::::::::::::::::::::::::::::::::::::::::::::::::::::::::::::::::::::::::%
\subsection{Tau distortion} 
%::::::::::::::::::::::::::::::::::::::::::::::::::::::::::::::::::::::::::::::::::::::::::::::::::::%

The harmonics transform of $\tau(\hatn)\T(\hatn)$ is 
%----------------------------------------------------------------------------------------------------%
\al{
	\delta\T_{\l m} &= \intn Y^*_{\l m}\tau (\hatn)\T(\hatn) 
		\notag \\ 
	&= \sum_{LM\l'm'}\tau_{LM}\T_{\l'm'} \intn Y^*_{\l m}Y_{LM}Y_{\l'm'}
		\notag \\ 
	&= \sum_{LM\l'm'}\tau_{LM}\T_{\l'm'} 
		p^+_{\l L\l'}\gamma_{\l L\l'} (-1)^m\Wjm{\l}{L}{\l'}{-m}{M}{m'}\Wjm{\l}{L}{\l'}{0}{0}{0}
		\notag \\ 
	&= \conv \tau_{LM} \T_{\l'm'} \W_{\l L\l'}^{\tau,0}
	\,, 
}
%----------------------------------------------------------------------------------------------------%
where 
%----------------------------------------------------------------------------------------------------%
\al{
	\W_{\l L\l'}^{\tau,0} = p^+_{\l L\l'}\gamma_{\l L\l'} \Wjm{\l}{L}{\l'}{0}{0}{0}  \,. 
}
%----------------------------------------------------------------------------------------------------%


%::::::::::::::::::::::::::::::::::::::::::::::::::::::::::::::::::::::::::::::::::::::::::::::::::::%
\subsection{Summary} 
%::::::::::::::::::::::::::::::::::::::::::::::::::::::::::::::::::::::::::::::::::::::::::::::::::::%

The above all distortions are described in the following form: 
%----------------------------------------------------------------------------------------------------%
\al{
	\delta \T_{\l m} &= \conv x_{LM} \T_{\l'm'} \W^{x,0}_{\l L\l'}
	\,, \\ 
	\delta E_{\l m}  &= \conv x_{LM} 
		\left( E_{\l'm'} \W^{x,+}_{\l L\l'} + B_{\l'm'} \W^{x,-}_{\l L\l'}\right) 
	\,, \\ 
	\delta B_{\l m}  &= \conv x_{LM}
		\left(-E_{\l'm'} \W^{x,-}_{\l L\l'} + B_{\l'm'} \W^{x,+}_{\l L\l'}\right) 
}
%----------------------------------------------------------------------------------------------------%
where $x$ is a distortion field. The functional form of $\W$ is given above. 

\newpage


%////////////////////////////////////////////////////////////////////////////////////////////////////%
\section{Quadratic estimators} 
%////////////////////////////////////////////////////////////////////////////////////////////////////%

%::::::::::::::::::::::::::::::::::::::::::::::::::::::::::::::::::::::::::::::::::::::::::::::::::::%
\subsection{Distortion induced anisotropies} 
%::::::::::::::::::::::::::::::::::::::::::::::::::::::::::::::::::::::::::::::::::::::::::::::::::::%

The distortion fields $x$ described above induce the off-diagonal elements of the covariance 
($\l\not=\l'$ or $m\not=m'$),
%----------------------------------------------------------------------------------------------------%
\al{
	\ave{\widetilde{X}_{\l m}\widetilde{Y}_{\l'm'}}\rom{CMB} 
		&= \sum_{LM}\Wjm{\l}{\l'}{L}{m}{m'}{M} f^{x,{\rm XY}}_{\l L \l'} x^*_{LM} 
	\,, \label{Eq:weight} 
}
%----------------------------------------------------------------------------------------------------%
where $\ave{\cdots}\rom{CMB}$ denotes the ensemble average over the primary CMB anisotropies 
with a fixed realization of the distortion fields. We ignore the higher-order terms 
of the distortion fields. The functional form of the weight functions $f$ are discussed later. 

With a quadratic combination of observed CMB anisotropies, $\hX$ and $\hY$, the general quadratic 
estimators are formed as (e.g., \cite{Okamoto:2003zw}), 
%----------------------------------------------------------------------------------------------------%
\al{
	[\estx^{\rm XY}_{LM}]^* = A^{x,{\rm XY}}_L \sum_{\l\l'mm'}\Wjm{\l}{\l'}{L}{m}{m'}{M}
		g^{x,{\rm XY}}_{\l L\l'}\hX_{\l m}\hY_{\l'm'}
	\,. \label{Eq:estg} 
}
%----------------------------------------------------------------------------------------------------%
Here we define 
%----------------------------------------------------------------------------------------------------%
\al{
	g^{x,{\rm XY}}_{\l L\l'} 
		&= \frac{[f^{x,{\rm XY}}_{\l L\l'}]^*}{\Delta^{\rm XY}\hCXX_\l\hCYY_{\l'}} 
	\\ 
	A^{x,{\rm XY}}_L  
		&= \frac{1}{2L+1}\sum_{\l\l'}f^{x,{\rm XY}}_{\l L\l'}g^{x,{\rm XY}}_{\l L\l'} 
	\,, \label{Eq:Rec:N0}
}
%----------------------------------------------------------------------------------------------------%
where $\Delta^{\rm XX}=2$, $\Delta^{\rm EB}=\Delta^{\rm TB}=1$, and $\hCXX_\l$ ($\hCYY_\l$) is 
the observed power spectrum. 

%::::::::::::::::::::::::::::::::::::::::::::::::::::::::::::::::::::::::::::::::::::::::::::::::::::%
\subsection{Weight Function} 
%::::::::::::::::::::::::::::::::::::::::::::::::::::::::::::::::::::::::::::::::::::::::::::::::::::%

The weight functions are, in general, given as
%----------------------------------------------------------------------------------------------------%
\al{
	f^{x,(\T\T)}_{\l L\l'} &= \W^{x,0}_{\l L\l'}\CTT_{\l'} + p_x \W^{x,0}_{\l'L\l}\CTT_\l \,, \\ 
	f^{x,(\T E)}_{\l L\l'} &= \W^{x,0}_{\l L\l'}\CTE_{\l'} + p_x \W^{x,+}_{\l'L\l}\CTE_\l \,, \\ 
	f^{x,(\T B)}_{\l L\l'} &= p_x \W^{x,-}_{\l'L\l}\CTE_\l \,, \\ 
	f^{x,(EE)}_{\l L\l'}   &= \W^{x,+}_{\l L\l'}\CEE_{\l'} + p_x \W^{x,+}_{\l'L\l}\CEE_\l \,, \\ 
	f^{x,(EB)}_{\l L\l'}   &= \W^{x,-}_{\l L\l'}\CBB_{\l'} + p_x \W^{x,-}_{\l'L\l}\CEE_\l \,, \\ 
	f^{x,(BB)}_{\l L\l'}   &= \W^{x,+}_{\l L\l'}\CBB_{\l'} + p_x \W^{x,+}_{\l'L\l}\CBB_\l \,. 
}
%----------------------------------------------------------------------------------------------------%
Here, the parity index is $p_\grad=p_\epsilon=1$ and $p_\curl=p_\alpha=-1$. 
Strictly speaking, $p_x$ should be $(-1)^{\l'+L+\l}$. However, $\W$ is only non-zero when $\l'+L+\l$ 
is even, and vice versa. The parity even quantities are $x=\grad$ and $\epsilon$. 
The odd parity quantities are $x=\curl$ and $\alpha$. 
Note that the above weight functions are consistent with Ref.~\cite{Namikawa:2011b} 
($\W^{x,-}_{\l L\l'}=-{}_{\ominus}S^x_{\l L\l'}$) for the lensing case. 

%::::::::::::::::::::::::::::::::::::::::::::::::::::::::::::::::::::::::::::::::::::::::::::::::::::%
\subsection{Weight Function: Derivations} 
%::::::::::::::::::::::::::::::::::::::::::::::::::::::::::::::::::::::::::::::::::::::::::::::::::::%

Let us first consider the temperature case. There are two contributions to the temperature quadratic 
estimator, and the one is given as
%----------------------------------------------------------------------------------------------------%
\al{
	\ave{\T_{\l''m''} \delta \T_{\l m}} 
		&= \conv x_{LM} \ave{\T_{\l''m''}\T_{\l'm'} }\W^{x,0}_{\l L\l'} \notag \\
		&= \conv x_{LM} \delta_{\l''\l'}\delta_{m'',-m'} (-1)^{m'} \CTT_{\l'}\W^{x,0}_{\l L\l'} \notag \\
		&= \sum_{LM} \Wjm{\l}{\l''}{L}{m}{m''}{M} x^*_{LM} \CTT_{\l''}\W^{x,0}_{\l L\l''}
	\,.
}
%----------------------------------------------------------------------------------------------------%
Here, we use
%----------------------------------------------------------------------------------------------------%
\al{
	\conv \delta_{\l''\l'}\delta_{m'',-m'} (-1)^{m'} x_{LM} 
		&= \sum_{LM} (-1)^{m+m'}\Wjm{\l}{L}{\l''}{-m}{M}{-m''} x_{LM} \notag \\
		&= \sum_{LM} (-1)^{-M} \Wjm{\l}{\l''}{L}{m}{m''}{-M} x_{LM} \notag \\
		&= \sum_{LM} \Wjm{\l}{\l''}{L}{m}{m''}{M} x^*_{LM}
}
%----------------------------------------------------------------------------------------------------%
The other term is obtained by exchanging $(\l'',m'')\leftrightarrow(\l,m)$ and is given by
%----------------------------------------------------------------------------------------------------%
\al{
	\ave{\T_{\l m} \delta \T_{\l''m''}} 
		&= \sum_{LM} (-1)^{\l+\l''+L}\Wjm{\l}{\l''}{L}{m}{m''}{M} x^*_{LM} \CTT_\l\W^{x,0}_{\l''L\l}
	\,.
}
%----------------------------------------------------------------------------------------------------%
The sign $(-1)^{\l+\l''+L}$ depends on the parity of $\W$. 

For $EB$ estimator, the two contributions are
%----------------------------------------------------------------------------------------------------%
\al{
	E_{\l''m''}\delta B_{\l m} 
		&= - \conv x_{LM} \ave{E_{\l''m''}E_{\l'm'}}\W^{x,-}_{\l L\l'} \notag \\
		&= - \conv x_{LM} (-1)^{m''}\delta_{\l''\l'}\delta_{m'',-m'}\CEE_{\l''}\W^{x,-}_{\l L\l'} \notag \\
		&= - \sum_{LM} \Wjm{\l}{\l''}{L}{m}{m''}{M} x^*_{LM} \CEE_{\l''}\W^{x,-}_{\l L\l'}
	\,,
}
%----------------------------------------------------------------------------------------------------%
and 
%----------------------------------------------------------------------------------------------------%
\al{
	B_{\l''m''}\delta E_{\l m} 
		&= \conv x_{LM} \ave{E_{\l''m''}E_{\l'm'}}\W^{x,-}_{\l L\l'} \notag \\
		&= \sum_{LM} (-1)^{\l+\l''+L} \Wjm{\l}{\l''}{L}{m}{m''}{M} x^*_{LM} \CEE_\l\W^{x,-}_{\l''L\l}
	\,. 
}
%----------------------------------------------------------------------------------------------------%


\newpage


%////////////////////////////////////////////////////////////////////////////////////////////////////%
\section{Fast Computation of Normalization} 
%////////////////////////////////////////////////////////////////////////////////////////////////////%

%::::::::::::::::::::::::::::::::::::::::::::::::::::::::::::::::::::::::::::::::::::::::::::::::::::%
\subsection{Normalization and Kernel function}
%::::::::::::::::::::::::::::::::::::::::::::::::::::::::::::::::::::::::::::::::::::::::::::::::::::%

The normalization of the $\T\T$ quadratic estimator is 
%----------------------------------------------------------------------------------------------------%
\al{
	\frac{1}{A_L^{x,(\T\T)}} 
		&= \frac{1}{2L+1}\sum_{\l\l'} \frac{\left[\W^{x,0}_{\l L\l'}\CTT_{\l'} 
			+ p_x \W^{x,0}_{\l'L\l}\CTT_\l\right]^2 }{2\hCTT_\l\hCTT_{\l'}} 
	\notag \\ 
		&= \frac{1}{2}\Sigma_L^{(0),x}\left[\frac{1}{\hCTT},\frac{(\CTT)^2}{\hCTT}\right] 
			+ p_x \Gamma_L^{(0),x}\left[\frac{\CTT}{\hCTT},\frac{\CTT}{\hCTT}\right]
			+ \frac{1}{2}\Sigma_L^{(0),x}\left[\frac{(\CTT)^2}{\hCTT},\frac{1}{\hCTT}\right]
	\,, 
}
%----------------------------------------------------------------------------------------------------%
where we define kernel functions as
%----------------------------------------------------------------------------------------------------%
\al{
	\Sigma_L^{(0),x}[A,B] &= \frac{1}{2L+1}\sum_{\l\l'}(\W^{x,0}_{\l L\l'})^2 A_\l B_{\l'} 
	\,, \\ 
	\Gamma_L^{(0),x}[A,B] &= \frac{1}{2L+1}\sum_{\l\l'} \W^{x,0}_{\l L\l'}\W^{x,0}_{\l'L\l} A_\l B_{\l'} 
	\,. 
}
%----------------------------------------------------------------------------------------------------%
For $\T E$, 
%----------------------------------------------------------------------------------------------------%
\al{
	\frac{1}{A_L^{x,(\T E)}} &= \frac{1}{2L+1}\sum_{\l\l'}
		\frac{|\W^{x,0}_{\l L\l'} \CTE_{\l'} + p_x \W^{x,+}_{\l'L\l}\CTE_\l|^2} {\hCTT_\l\hCEE_{\l'}}
	\notag \\ 
	&= \frac{1}{2L+1}\sum_{\l\l'} \left[(\W^{x,0}_{\l L\l'})^2\frac{(\CTE_{\l'})^2}{\hCTT_\l\hCEE_{\l'}} 
		+ 2p_x \W^{x,0}_{\l L\l'} \W^{x,+}_{\l'L\l} 
		\frac{\CTE_{\l'}\CTE_\l}{\hCTT_\l\hCEE_{\l'}} + (\W^{x,+}_{\l'L\l})^2
		\frac{(\CTE_\l)^2}{\hCTT_\l\hCEE_{\l'}} \right]
	\notag \\ 
	&= \Sigma_L^{(0),x} \left[\frac{1}{\hCTT},\frac{(\CTE)^2}{\hCEE}\right] 
		+ 2p_x \Gamma_L^{(\times ),x} \left[\frac{\CTE}{\hCTT},\frac{\CTE}{\hCEE}\right]
		+ \Sigma_L^{(+),x} \left[\frac{1}{\hCEE},\frac{(\CTE)^2}{\hCTT}\right]
	\,, 
}
%----------------------------------------------------------------------------------------------------%
where kernel functions are defined as
%----------------------------------------------------------------------------------------------------%
\al{
	\Gamma_L^{(\times ),x}[A,B] &= \frac{1}{2L+1}\sum_{\l\l'} \W^{x,0}_{\l L\l'} \W^{x,+}_{\l'L\l}A_\l B_{\l'}
	\,, \notag \\ 
	\Sigma_L^{(+),x}[A,B] &= \frac{1}{2L+1}\sum_{\l\l'} (\W^{x,+}_{\l L\l'})^2 A_\l B_{\l'}
	\,. 
}
%----------------------------------------------------------------------------------------------------%
For $\T B$, 
%----------------------------------------------------------------------------------------------------%
\al{
	\frac{1}{A_L^{x,(\T B)}} &= \frac{1}{2L+1}\sum_{\l\l'}
		\frac{|\W^{x,-}_{\l'L\l}\CTE_\l|^2}{\hCTT_\l\hCBB_{\l'}}
	\notag \\ 
	&= \Sigma_L^{(-),x}\left[\frac{1}{\hCBB},\frac{(\CTE)^2}{\hCTT}\right] 
	\,, 
}
%----------------------------------------------------------------------------------------------------%
where 
%----------------------------------------------------------------------------------------------------%
\al{
	\Sigma_L^{(-),x}[A,B] &= \frac{1}{2L+1}\sum_{\l\l'}|\W^{x,-}_{\l L\l'}|^2 A_\l B_{\l'} \,. 
}
%----------------------------------------------------------------------------------------------------%
For EE (and for BB by replacing EE $\to$ BB), 
%----------------------------------------------------------------------------------------------------%
\al{
	\frac{1}{A_L^x} &= \frac{1}{2L+1}\sum_{\l\l'} \frac{|\W^{x,+}_{\l L\l'}\CEE_{\l'} 
		+ p_x \W^{x,+}_{\l'L\l}\CEE_\l|^2}{2\hCEE_\l\hCEE_{\l'}}
	\notag \\ 
	&= \Sigma_L^{(+),x}\left[\frac{1}{\hCEE},\frac{(\CEE)^2}{\hCEE}\right] 
		+ p_x \Gamma_L^{(+),x} \left[\frac{\CEE}{\hCEE},\frac{\CEE}{\hCEE}\right]
	\,, 
}
%----------------------------------------------------------------------------------------------------%
where 
%----------------------------------------------------------------------------------------------------%
\al{
	\Gamma_L^{(+),x}[A,B] &= \frac{1}{2L+1}\sum_{\l\l'} \W^{x,+}_{\l'L\l} \W^{x,+}_{\l L\l'}A_\l B_{\l'} 
		= \Gamma_L^{(+),x}[B,A]
	\,. 
}
%----------------------------------------------------------------------------------------------------%
For $EB$, 
%----------------------------------------------------------------------------------------------------%
\al{
	\frac{1}{A_L^{x,(EB)}} &= \frac{1}{2L+1}\sum_{\l\l'} \frac{|\W^{x,-}_{\l L\l'}\CBB_{\l'} 
		+ p_x \W^{x,-}_{\l'L\l}\CEE_\l|^2}{\hCEE_\l\hCBB_{\l'}}
	\notag \\ 
	&= \Sigma_L^{(-),x} \left[\frac{1}{\hCEE},\frac{(\CBB)^2}{\hCBB}\right] 
		+ 2p_x \Gamma_L^{(-),x} \left[\frac{\CEE}{\hCEE},\frac{\CBB}{\hCBB}\right]
		+ \Sigma_L^{(-),x} \left[\frac{1}{\hCBB},\frac{(\CEE)^2}{\hCEE}\right]
	\,, 
}
%----------------------------------------------------------------------------------------------------%
where 
%----------------------------------------------------------------------------------------------------%
\al{
	\Gamma_L^{(-),x}[A,B] &= \frac{1}{2L+1}\sum_{\l\l'}
		[\W^{x,-}_{\l L\l'}]^* \W^{x,-}_{\l'L\l}A_\l B_{\l'} = \Gamma_L^{(-),x}[B,A] \,. 
}
%----------------------------------------------------------------------------------------------------%

%::::::::::::::::::::::::::::::::::::::::::::::::::::::::::::::::::::::::::::::::::::::::::::::::::::%
\subsection{Noise covariance and kernel function}
%::::::::::::::::::::::::::::::::::::::::::::::::::::::::::::::::::::::::::::::::::::::::::::::::::::%

For $\T\T\T E$, 
%----------------------------------------------------------------------------------------------------%
\al{
	\frac{A_L^{x,(\T\T)}A_L^{x,(\T E)}}{N_L^{x,(\T\T\T E)}} 
	&= \frac{1}{2L+1}\sum_{\l\l'} \left[\frac{\W^{x,0}_{\l L\l'}\CTT_{\l'}}{2\hCTT_\l\hCTT_{\l'}} 
		+ p_x (\l\leftrightarrow \l')\right] \left[ \frac{(\W^{x,0}_{\l L\l'}\CTE_{\l'} 
		+ p_x \W^{x,+}_{\l'L\l}\CTE_\l)\hCTE_{\l'}}{\hCEE_{\l'}}
		+ p_x (\l\leftrightarrow\l') \right]
	\notag \\ 
	&= \frac{1}{2L+1}\sum_{\l\l'} \bigg[\frac{\W^{x,0}_{\l L\l'}\CTT_{\l'}}{\hCTT_\l\hCTT_{\l'}} 
		\frac{(\W^{x,0}_{\l L\l'}\CTE_{\l'} + p_x \W^{x,+}_{\l'L\l}\CTE_\l)\hCTE_{\l'}}{\hCEE_{\l'}}
	\notag \\ 
	&\qquad + p_x \frac{\W^{x,0}_{\l L\l'}\CTT_{\l'}}{\hCTT_\l\hCTT_{\l'}}
		\frac{(\W^{x,0}_{\l'L\l}\CTE_\l + p_x \W^{x,+}_{\l L\l'}\CTE_{\l'})\hCTE_\l}{\hCEE_\l} \bigg]
	\notag \\ 
	&= \Sigma_L^{(0),x} \left[\frac{1}{\hCTT},\frac{\CTT\CTE\hCTE}{\hCTT\hCEE}\right] 
		+ p_x \Gamma_L^{(\times),x} \left[\frac{\CTE}{\hCTT},\frac{\CTT\hCTE}{\hCTT\hCEE}\right] 
	\notag \\ 
	&\qquad + p_x \Gamma_L^{(0),x} \left[\frac{\CTE\hCTE}{\hCTT\hCEE},\frac{\CTT}{\hCTT}\right] 
		+ \Sigma_L^{(\times),x} \left[\frac{\hCTE}{\hCTT\hCEE},\frac{\CTE\CTT}{\hCTT}\right] 
	\,, 
}
%----------------------------------------------------------------------------------------------------%
where 
%----------------------------------------------------------------------------------------------------%
\al{
	\Sigma_L^{(\times),x}[A,B] = \frac{1}{2L+1}\sum_{\l\l'} \W^{x,0}_{\l L\l'}\W^{x,+}_{\l L\l'} A_\l B_{\l'} 
	\,. 
}
%----------------------------------------------------------------------------------------------------%
For $\T\T EE$, 
%----------------------------------------------------------------------------------------------------%
\al{
	\frac{A_L^{x,(\T\T)}A_L^{x,(EE)}}{N_L^{x,(\T\T EE)}} 
	&= \frac{1}{2L+1}\sum_{\l\l'} \left[ \frac{\W^{x,0}_{\l L\l'}\CTT_{\l'}}{2\hCTT_\l\hCTT_{\l'}} 
		+ p_x (\l\leftrightarrow \l')\right]
		\left[ \frac{(\W^{x,+}_{\l L\l'}\CEE_{\l'} + p_x \W^{x,+}_{\l' L\l}\CEE_\l) \hCTE_\l\hCTE_{\l'}}
		{2\hCEE_\l\hCEE_{\l'}} + p_x (\l\leftrightarrow\l') \right] 
		\notag \\ 
	&= \frac{1}{2L+1}\sum_{\l\l'} \frac{\W^{x,0}_{\l L\l'}\CTT_{\l'}}{\hCTT_\l\hCTT_{\l'}} 
		\left[ \frac{(\W^{x,+}_{\l L\l'}\CEE_{\l'} 
		+ p_x \W^{x,+}_{\l'L\l}\CEE_\l) \hCTE_\l\hCTE_{\l'}} {2\hCEE_\l\hCEE_{\l'}} 
		+ p_x (\l\leftrightarrow\l') \right]
		\notag \\ 
	&= \frac{1}{2L+1}\sum_{\l\l'} \frac{\W^{x,0}_{\l L\l'}\CTT_{\l'}}{\hCTT_\l\hCTT_{\l'}} 
		\frac{(\W^{x,+}_{\l L\l'}\CEE_{\l'} + p_x \W^{x,+}_{\l'L\l}\CEE_\l)\hCTE_\l\hCTE_{\l'}}{\hCEE_\l\hCEE_{\l'}} 
		\notag \\ 
	&= \Sigma_L^{(0),x}\left[\frac{\hCTE}{\hCTT\hCEE},\frac{\CTT\CEE\hCTE}{\hCTT\hCEE}\right] 
		+ p_x \Gamma_L^{(\times),x}\left[\frac{\hCTE\CEE}{\hCTT\hCEE},\frac{\CTT\hCTE}{\hCTT\hCEE}\right] 
	\,. 
}
%----------------------------------------------------------------------------------------------------%
For $\T EEE$, 
%----------------------------------------------------------------------------------------------------%
\al{
	\frac{A_L^{x,(\T E)}A_L^{x,(EE)}}{N_L^{x,(\T EEE)}} &= \frac{1}{2L+1}\sum_{\l\l'}
		\left[ \frac{\W^{x,+}_{\l L\l'}\CEE_{\l'}}{2\hCEE_\l\hCEE_{\l'}} 
			+ p_x (\l\leftrightarrow \l') \right]
		\left[ \frac{(\W^{x,0}_{\l L\l'}\CTE_{\l'} 
			+ p_x \W^{x,+}_{\l'L\l}\CTE_\l)\hCTE_\l}{\hCTT_\l} + p_x (\l\leftrightarrow \l') \right]
	\notag \\ 
	&= \frac{1}{2L+1}\sum_{\l\l'} \left[ \frac{\W^{x,+}_{\l L\l'}\CEE_{\l'}}{\hCEE_\l\hCEE_{\l'}} 
			+ p_x \frac{\W^{x,+}_{\l'L\l}\CEE_\l}{\hCEE_\l\hCEE_{\l'}} \right]
		\left[ \frac{(\W^{x,0}_{\l L\l'}\CTE_{\l'} 
			+ p_x \W^{x,+}_{\l'L\l}\CTE_\l)\hCTE_\l}{\hCTT_\l} \right]
	\notag \\ 
	&= \Sigma_L^{(\times),x}\left[\frac{\hCTE}{\hCTT\hCEE},\frac{\CTE\CEE}{\hCEE}\right] 
		+ p_x \Gamma_L^{(+),x} \left[\frac{\CTE\hCTE}{\hCTT\hCEE},\frac{\CEE}{\hCEE}\right] 
	\notag \\ 
	&\qquad + p_x \Gamma_L^{(\times),x} \left[\frac{\hCTE\CEE}{\hCTT\hCEE},\frac{\CTE}{\hCEE}\right] 
		+ \Sigma_L^{(+),x} \left[\frac{\CTE\hCTE\CEE}{\hCTT\hCEE},\frac{1}{\hCEE}\right] 
	\,. 
}
%----------------------------------------------------------------------------------------------------%
For $\T BEB$, 
%----------------------------------------------------------------------------------------------------%
\al{
	\frac{A_L^{x,(\T B)}A_L^{x,(EB)}}{N_L^{x,(\T BEB)}} &= \frac{1}{2L+1}\sum_{\l\l'}
		\left[ \frac{(\W^{x,-}_{\l L\l'})^*\CBB_{\l'} 
			- p_x (\W^{x,-}_{\l'L\l})^*\CEE_\l}{\hCEE_\l\hCBB_{\l'}} \right]
		\left[ \frac{-p_x \W^{x,-}_{\l'L\l}\CTE_\l\hCTE_\l}{\hCTT_\l} \right]
	\notag \\ 
	&= - p_x \Gamma_L^{(-),x} \left[\frac{\CTE\hCTE}{\hCTT\hCEE},\frac{\CBB}{\hCBB}\right] 
		+ \Sigma_L^{(-),x} \left[\frac{\CTE\hCTE\CEE}{\hCTT\hCEE},\frac{1}{\hCBB}\right] 
	\,. 
}
%----------------------------------------------------------------------------------------------------%


\newpage

%////////////////////////////////////////////////////////////////////////////////////////////////////%
\section{Explicit Kernel Functions}
%////////////////////////////////////////////////////////////////////////////////////////////////////%

Here we consider expression for the Kernel functions in terms of the Wigner d-functions. 
In the following calculations, we frequently use 
%----------------------------------------------------------------------------------------------------%
\al{
	\INT{}{\mu}{}{-1}{1} d^{\l_1}_{s_1,s'_1}(\beta) d^{\l_2}_{s_2,s'_2}(\beta) d^{\l_3}_{s_3,s'_3}(\beta) 
		= 2 \Wjm{\l_1}{\l_2}{\l_3}{s_1}{s_2}{s_3} \Wjm{\l_1}{\l_2}{\l_3}{s'_1}{s'_2}{s'_3}
	\,, 
}
%----------------------------------------------------------------------------------------------------%
with $s_1+s_2+s_3=s'_1+s'_2+s'_3=0$ and $\mu=\cos\beta$, and the symmetric property: 
%----------------------------------------------------------------------------------------------------%
\al{
	d_{mm'}^\l(\beta) &= (-1)^{m-m'} d_{-m,-m'}^\l(\beta) = (-1)^{m-m'}d_{m'm}^\l(\beta)  \\ 
	d_{mm'}^\l(\beta) &= (-1)^{\l+m} d_{m,-m'}^\l(\pi-\beta) \,. 
}
%----------------------------------------------------------------------------------------------------%
We also define 
%----------------------------------------------------------------------------------------------------%
\al{
	X^{p\dots q} &= a_\l^p\dots a_\l^q X_\l \,. 
}
%----------------------------------------------------------------------------------------------------%

%::::::::::::::::::::::::::::::::::::::::::::::::::::::::::::::::::::::::::::::::::::::::::::::::::::%
\subsection{Kernel Functions: Lensing}
%::::::::::::::::::::::::::::::::::::::::::::::::::::::::::::::::::::::::::::::::::::::::::::::::::::%

We obtain 
%----------------------------------------------------------------------------------------------------%
\al{
	\Sigma_L^{(0),x}[A,B] &= \frac{1}{2L+1}\sum_{\l\l'} |\W^{x,0}_{\l L\l'}|^2 A_\l B_{\l'}
	\notag \\ 
		&= \sum_{\l\l'}4\pi L(L+1)\frac{2\l+1}{4\pi}A_\l\frac{2\l'+1}{4\pi}B_{\l'}
		\l'(\l'+1) \frac{1+c_x^2(-1)^{\l+L+\l'}}{2}\Wjm{\l}{L}{\l'}{0}{1}{-1}^2 
	\notag \\ 
		&= \pi L(L+1)\sum_{\l\l'}\frac{2\l+1}{4\pi}A_\l\frac{2\l'+1}{4\pi}B_{\l'}
		2\l'(\l'+1) \left[\Wjm{\l}{L}{\l'}{0}{1}{-1}^2 
		+ c_x^2 \Wjm{\l}{L}{\l'}{0}{-1}{1}\Wjm{\l}{L}{\l'}{0}{1}{-1}\right]
	\notag \\ 
		&= \INT{}{\mu}{}{-1}{1} \pi L(L+1)\sum_{\l\l'}
			\frac{2\l+1}{4\pi}A_\l\frac{2\l'+1}{4\pi}B_{\l'} \l'(\l'+1) [d^\l_{00}d^L_{11}d^{\l'}_{11}
			+ c_x^2 d^\l_{00}d^L_{1,-1}d^{\l'}_{1,-1} ]
	\notag \\ 
		&= \INT{}{\mu}{}{-1}{1} \pi L(L+1)
		\{\xi_{00}[A]\xi_{11}[B^{00}]d^L_{11}+c_x^2\xi_{00}[A]\xi_{1,-1}[B^{00}]d^L_{1,-1}\}
	\,, 
}
%----------------------------------------------------------------------------------------------------%
where 
%----------------------------------------------------------------------------------------------------%
\al{
	\xi_{mm'}[A] &= \sum_\l\frac{2\l+1}{4\pi}A_\l d^\l_{mm'} \,. 
}
%----------------------------------------------------------------------------------------------------%
The cross-term is 
%----------------------------------------------------------------------------------------------------%
\al{
	\Gamma_L^{(0),x}[A,B] &= \frac{1}{2L+1}\sum_{\l\l'}
		(\W^{x,0}_{\l L\l'})^*\W^{x,0}_{\l'L\l} A_\l B_{\l'}
	\notag \\ 
		&= \sum_{\l\l'}4\pi L(L+1)\frac{2\l+1}{4\pi}A_\l\frac{2\l'+1}{4\pi}B_{\l'}
		a_\l^0 a_{\l'}^0 \frac{1+c_x^2(-1)^{\l+L+\l'}}{2}
		\Wjm{\l}{L}{\l'}{0}{1}{-1}\Wjm{\l'}{L}{\l}{0}{1}{-1} 
	\notag \\ 
		&= \pi L(L+1)\sum_{\l\l'}\frac{2\l+1}{4\pi}A^0_\l\frac{2\l'+1}{4\pi}B^0_{\l'} 
		2\left[\Wjm{\l}{L}{\l'}{0}{1}{-1}\Wjm{\l}{L}{\l'}{1}{-1}{0}
			+ c_x^2 \Wjm{\l}{L}{\l'}{0}{1}{-1}\Wjm{\l}{L}{\l'}{-1}{1}{0}\right]
	\notag \\ 
		&= \INT{}{\mu}{}{-1}{1} \pi L(L+1)\sum_{\l\l'}
			\frac{2\l+1}{4\pi}A^0_\l\frac{2\l'+1}{4\pi}B^0_{\l'}
			[d^\l_{01}d^L_{1,-1}d^{\l'}_{-1,0} + c_x^2 d^\l_{0,-1}d^L_{11}d^{\l'}_{-1,0} ]
	\notag \\ 
		&= - \INT{}{\mu}{}{-1}{1} \pi L(L+1)
		\{\xi_{01}[A^0]\xi_{0,-1}[B^0]d^L_{1,-1}+c_x^2\xi_{01}[A^0]\xi_{01}[B^0]d^L_{11}\}
	\,. 
}
%----------------------------------------------------------------------------------------------------%

Denoting $p=\pm$ and $x=\grad,\curl$, the polarization auto kernel is 
%----------------------------------------------------------------------------------------------------%
\al{
	&\Sigma_L^{(p),x}[A,B] = \frac{1}{2L+1}\sum_{\l\l'}|\W^{x,p}_{\l L\l'}|^2 A_\l B_{\l'}
	\notag \\ 
	&= \pi L(L+1)\sum_{\l\l'}\frac{2\l+1}{4\pi}A_\l\frac{2\l'+1}{4\pi}B_{\l'}
		2[1+pc_x^2 (-1)^{\l+L+\l'}] \left[a_{\l'}^+ \Wjm{\l}{L}{\l'}{-2}{-1}{3} 
		+ c_x^2 a_{\l'}^- \Wjm{\l}{L}{\l'}{-2}{1}{1}\right]^2
	\notag \\
	&= \pi L(L+1)\sum_{\l\l'}\frac{2\l+1}{4\pi}A_\l\frac{2\l'+1}{4\pi}B_{\l'}[1+pc_x^2 (-1)^{\l+L+\l'}] 
	\notag \\
	&\quad \times 2\bigg[ (a_{\l'}^+)^2 \Wjm{\l}{L}{\l'}{-2}{-1}{3}^2
		+(a_{\l'}^-)^2\Wjm{\l}{L}{\l'}{-2}{1}{1}^2 + 2c_x^2 a_{\l'}^+a_{\l'}^- 
		\Wjm{\l}{L}{\l'}{-2}{1}{1}\Wjm{\l}{L}{\l'}{-2}{-1}{3}\bigg] 
	\notag \\
	&= \pi L(L+1)\sum_{\l\l'}\frac{2\l+1}{4\pi}A_\l\frac{2\l'+1}{4\pi}B_{\l'}
	\notag \\
	&\quad \times 2\bigg[ (a_{\l'}^+)^2 \Wjm{\l}{L}{\l'}{-2}{-1}{3}^2
		+ (a_{\l'}^-)^2\Wjm{\l}{L}{\l'}{-2}{1}{1}^2 + 2c_x^2 a_{\l'}^+a_{\l'}^- 
		\Wjm{\l}{L}{\l'}{-2}{1}{1}\Wjm{\l}{L}{\l'}{-2}{-1}{3}
	\notag \\ 
	&\quad + pc_x^2 (a_{\l'}^+)^2\Wjm{\l}{L}{\l'}{-2}{-1}{3}\Wjm{\l}{L}{\l'}{2}{1}{-3}
		+ pc_x^2 (a_{\l'}^-)^2\Wjm{\l}{L}{\l'}{-2}{1}{1}\Wjm{\l}{L}{\l'}{2}{-1}{-1} 
		+ 2p a_{\l'}^+a_{\l'}^- \Wjm{\l}{L}{\l'}{-2}{1}{1}\Wjm{\l}{L}{\l'}{2}{1}{-3}\bigg]
	\notag \\
	&= \INT{}{\mu}{}{-1}{1} \pi L(L+1)\sum_{\l\l'} \frac{2\l+1}{4\pi}A_\l\frac{2\l'+1}{4\pi}B_{\l'}
		[(a_{\l'}^+)^2 d^\l_{22}d^L_{11}d^{\l'}_{33} + (a_{\l'}^-)^2 d^\l_{22}d^L_{11}d^{\l'}_{11} 
	\notag \\
	&\quad + 2c_x^2 a_{\l'}^+a_{\l'}^- d^\l_{22}d^L_{1,-1}d^{\l'}_{13}
		+ pc_x^2 (a_{\l'}^+)^2 d^\l_{-2,2}d^L_{-1,1}d^{\l'}_{3,-3}
		+ pc_x^2 (a_{\l'}^-)^2 d^\l_{-2,2}d^L_{1,-1}d^{\l'}_{1,-1}
		+ 2 pa_{\l'}^+a_{\l'}^- d^\l_{-2,2}d^L_{11}d^{\l'}_{1,-3}]
	\notag \\ 
	&= \INT{}{\mu}{}{-1}{1} \pi L(L+1)[(\xi_{22}[A]\xi_{33}[B^{++}] + \xi_{22}[A]\xi_{11}[B^{--}]
		+ 2p \xi_{2,-2}[A]\xi_{3,-1}[B^{+-}])d^L_{11} 
	\notag \\ 
	&\qquad + c_x^2(p\xi_{2,-2}[A]\xi_{3,-3}[B^{++}] + p\xi_{2,-2}[A]\xi_{1,-1}[B^{--}] 
		+ 2\xi_{22}[A]\xi_{31}[B^{+-}])d^L_{1,-1}] 
	\,. 
}
%----------------------------------------------------------------------------------------------------%

The polarization cross kernel is 
%----------------------------------------------------------------------------------------------------%
\al{
	& \Gamma_L^{(p),x}[A,B] = \frac{1}{2L+1}\sum_{\l\l'} 
		(\W^{x,p}_{\l L\l'})^* \W^{x,p}_{\l'L\l} A_\l B_{\l'}
	\notag \\
	&= \pi L(L+1)\sum_{\l\l'}\frac{2\l+1}{4\pi}A_\l\frac{2\l'+1}{4\pi}B_{\l'}
		2[1+pc_x^2 (-1)^{\l+L+\l'}] 
	\notag \\
	&\quad\times \left[a_{\l'}^+ \Wjm{\l}{L}{\l'}{-2}{-1}{3} 
		+ c_x^2 a_{\l'}^- \Wjm{\l}{L}{\l'}{-2}{1}{1}\right]
		\left[a_\l^+ \Wjm{\l'}{L}{\l}{-2}{-1}{3} + c_x^2 a_\l^- \Wjm{\l'}{L}{\l}{-2}{1}{1}\right]
	\notag \\
	&= \pi L(L+1)\sum_{\l\l'}\frac{2\l+1}{4\pi}A_\l\frac{2\l'+1}{4\pi}B_{\l'}
		2[(-1)^{\l+L+\l'}+pc_x^2] 
	\notag \\
	&\quad\times \left[a_{\l'}^+ \Wjm{\l}{L}{\l'}{-2}{-1}{3} 
		+ c_x^2 a_{\l'}^- \Wjm{\l}{L}{\l'}{-2}{1}{1}\right]
		\left[a_\l^+ \Wjm{\l}{L}{\l'}{3}{-1}{-2} + c_x^2 a_\l^- \Wjm{\l}{L}{\l'}{1}{1}{-2}\right]
	\notag \\
	&= \pi L(L+1)\sum_{\l\l'}\frac{2\l+1}{4\pi}A_\l\frac{2\l'+1}{4\pi}B_{\l'} 
	\notag \\
	&\quad\times 2\bigg\{\left[a_{\l'}^+ \Wjm{\l}{L}{\l'}{2}{1}{-3} 
		+ c_x^2 a_{\l'}^- \Wjm{\l}{L}{\l'}{2}{-1}{-1}\right]
		\left[a_\l^+ \Wjm{\l}{L}{\l'}{3}{-1}{-2} + c_x^2 a_\l^- \Wjm{\l}{L}{\l'}{1}{1}{-2}\right]
	\notag \\ 
	&\quad + p\left[c_x^2 a_{\l'}^+ \Wjm{\l}{L}{\l'}{-2}{-1}{3} 
		+ a_{\l'}^- \Wjm{\l}{L}{\l'}{-2}{1}{1}\right]
		\left[a_\l^+ \Wjm{\l}{L}{\l'}{3}{-1}{-2} + c_x^2 a_\l^- \Wjm{\l}{L}{\l'}{1}{1}{-2}\right] \bigg\}
	\notag \\
	&= \INT{}{\mu}{}{-1}{1}\pi L(L+1)\sum_{\l\l'}\frac{2\l+1}{4\pi}A_\l\frac{2\l'+1}{4\pi}B_{\l'} 
	\notag \\
	&\quad\times [a_{\l'}^+a_\l^+ d^\l_{23}d^L_{1,-1}d^{\l'}_{-3,-2}
		+ c_x^2 a_{\l'}^+ a_\l^- d^\l_{21}d^L_{11}d^{\l'}_{-3,-2}
		+ c_x^2 a_{\l'}^- a_\l^+ d^\l_{23}d^L_{11}d^{\l'}_{-1,-2}
		+ a_{\l'}^- a_\l^- d^\l_{21}d^L_{1,-1}d^{\l'}_{-1,-2}
	\notag \\ 
	&\quad + p(c_x^2 a_{\l'}^+a_\l^+ d^\l_{-2,3}d^L_{11}d^{\l'}_{3,-2}
		+ a_{\l'}^+ a_\l^- d^\l_{-2,1}d^L_{1,-1}d^{\l'}_{3,-2}
		+ a_{\l'}^- a_\l^+ d^\l_{-2,3}d^L_{1,-1}d^{\l'}_{1,-2}
		+ c_x^2 a_{\l'}^- a_\l^- d^\l_{-2,1}d^L_{11}d^{\l'}_{1,-2})]
	\notag \\ 
	&= \INT{}{\mu}{}{-1}{1} \pi L(L+1)
		[-c_x^2(\xi_{21}[A^-]\xi_{32}[B^+]+\xi_{32}[A^+]\xi_{21}[B^-] 
		+ p\xi_{3,-2}[A^+]\xi_{3,-2}[B^+]+p\xi_{2,-1}[A^-]\xi_{2,-1}[B^-] )d^L_{11} 
	\notag \\ 
	&\qquad + (\xi_{32}[A^+]\xi_{32}[B^+]+\xi_{21}[A^-]\xi_{21}[B^-]
		- p\xi_{2,-1}[A^-]\xi_{3,-2}[B^+] - p\xi_{3,-2}[A^+]\xi_{2,-1}[B^-])d^L_{1,-1}] 
	\,. 
}
%----------------------------------------------------------------------------------------------------%

The temperature-polarization kernel is 
%----------------------------------------------------------------------------------------------------%
\al{
	\Sigma_L^{(\times),x}[A,B] &= \frac{1}{2L+1}\sum_{\l\l'}
		(\W^{x,0}_{\l L\l'})^*\W^{x,+}_{\l L\l'} A_\l B_{\l'}
		\notag \\ 
	&= \pi L(L+1)\sum_{\l\l'}\frac{2\l+1}{4\pi}A_\l\frac{2\l'+1}{4\pi}B_{\l'}
		a_{\l'}^0 2[1+c_x^2 (-1)^{\l+L+\l'}] 
		\notag \\ 
	&\qquad \times 
		\Wjm{\l}{L}{\l'}{0}{1}{-1}\left[a_{\l'}^+\Wjm{\l}{L}{\l'}{-2}{-1}{3}
		+ c_x^2 a_{\l'}^-\Wjm{\l}{L}{\l'}{-2}{1}{1}\right]
		\notag \\ 
	&= \pi L(L+1)\sum_{\l\l'}\frac{2\l+1}{4\pi}A_\l\frac{2\l'+1}{4\pi}B_{\l'}a_{\l'}^0  
		\notag \\ 
	&\qquad \times 
		2\left[\Wjm{\l}{L}{\l'}{0}{1}{-1}+c_x^2\Wjm{\l}{L}{\l'}{0}{-1}{1}\right]
		\left[a_{\l'}^+\Wjm{\l}{L}{\l'}{-2}{-1}{3} + c_x^2 a_{\l'}^-\Wjm{\l}{L}{\l'}{-2}{1}{1}\right]
		\notag \\ 
	&= \INT{}{\mu}{}{-1}{1} \pi L(L+1)\sum_{\l\l'}
		\frac{2\l+1}{4\pi}A_\l\frac{2\l'+1}{4\pi}B_{\l'}a_{\l'}^0
		\notag \\
	&\qquad \times \bigg[
		a_{\l'}^+ d^\l_{0,-2}d^L_{1,-1}d^{\l'}_{-1,3} + c_x^2 a_{\l'}^- d^\l_{0,-2}d^L_{11}d^{\l'}_{-1,1}
		+ c_x^2 a_{\l'}^+ d^\l_{0,-2}d^L_{11}d^{\l'}_{13} + a_{\l'}^- d^\l_{0,-2}d^L_{-1,1}d^{\l'}_{11} 
	\bigg] 
		\notag \\ 
	&= \INT{}{\mu}{}{-1}{1} \pi L(L+1)
		\{ c_x^2(\xi_{20}[A]\xi_{1,-1}[B^{0-}]+\xi_{20}[A]\xi_{31}[B^{0+}] )d^L_{11} 
		\notag \\
	&\qquad +(\xi_{20}[A]\xi_{3,-1}[B^{0+}]+\xi_{20}[A]\xi_{11}[B^{0-}])d^L_{1,-1}\}
	\,, 
}
%----------------------------------------------------------------------------------------------------%
and 
%----------------------------------------------------------------------------------------------------%
\al{
	\Gamma_L^{(\times),x}[A,B] &= \frac{1}{2L+1}\sum_{\l\l'}
		(\W^{x,0}_{\l L\l'})^* \W^{x,+}_{\l'L\l}A_\l B_{\l'}
		\notag \\ 
	&= \pi L(L+1)\sum_{\l\l'}\frac{2\l+1}{4\pi}A_\l\frac{2\l'+1}{4\pi}B_{\l'}
		a_{\l'}^0 2[1+c_x^2 (-1)^{\l+L+\l'}] 
		\notag \\ 
	&\qquad \times \Wjm{\l}{L}{\l'}{0}{1}{-1}\left[a_\l^+\Wjm{\l'}{L}{\l}{-2}{-1}{3}
		+ c_x^2 a_\l^-\Wjm{\l'}{L}{\l}{-2}{1}{1}\right]
		\notag \\ 
	&= \pi L(L+1)\sum_{\l\l'}\frac{2\l+1}{4\pi}A_\l\frac{2\l'+1}{4\pi}B_{\l'} a_{\l'}^0 
		\notag \\ 
	&\qquad \times 2\left[\Wjm{\l}{L}{\l'}{0}{1}{-1}+c_x^2\Wjm{\l}{L}{\l'}{0}{-1}{1}\right]
		\left[a_\l^+\Wjm{\l}{L}{\l'}{-3}{1}{2} + c_x^2 a_\l^-\Wjm{\l}{L}{\l'}{-1}{-1}{2}\right]
		\notag \\ 
	&= \INT{}{\mu}{}{-1}{1} \pi L(L+1)\sum_{\l\l'}
		\frac{2\l+1}{4\pi}A_\l\frac{2\l'+1}{4\pi}B_{\l'}a_{\l'}^0
		\notag \\
	&\qquad \times \bigg[
		a_\l^+ d^\l_{0,-3}d^L_{11}d^{\l'}_{-1,2} + c_x^2 a_\l^- d^\l_{0,-1}d^L_{1,-1}d^{\l'}_{-1,2}
		+ c_x^2 a_\l^+ d^\l_{0,-3}d^L_{1,-1}d^{\l'}_{12} + a_\l^- d^\l_{0,-1}d^L_{11}d^{\l'}_{12} 
		\bigg] 
		\notag \\ 
	&= \INT{}{\mu}{}{-1}{1} \pi L(L+1)
		\{ -(\xi_{30}[A^+]\xi_{2,-1}[B^0]+\xi_{10}[A^-]\xi_{12}[B^0] )d^L_{11} 
		\notag \\
	&\qquad - c_x^2(\xi_{10}[A^-]\xi_{2,-1}[B^0]+\xi_{30}[A^0]\xi_{21}[B^-])d^L_{1,-1}\}
	\,. 
}
%----------------------------------------------------------------------------------------------------%

%::::::::::::::::::::::::::::::::::::::::::::::::::::::::::::::::::::::::::::::::::::::::::::::::::::%
\subsection{Kernel Functions: Rotation}
%::::::::::::::::::::::::::::::::::::::::::::::::::::::::::::::::::::::::::::::::::::::::::::::::::::%

Next we consider the kernel functions for $x=\alpha$. If $p=-$ and $x=\alpha$, 
%----------------------------------------------------------------------------------------------------%
\al{
	\Sigma_L^{(-),\alpha}[A,B] 
	&= \pi \sum_{\l\l'}\frac{2\l+1}{4\pi}A_\l\frac{2\l'+1}{4\pi}B_{\l'}
		8[1+(-1)^{\l+L+\l'}] \Wjm{\l}{L}{\l'}{-2}{0}{2}^2
	\notag \\
	&= \pi L(L+1)\sum_{\l\l'}\frac{2\l+1}{4\pi}A_\l\frac{2\l'+1}{4\pi}B_{\l'}
		8\bigg[ \Wjm{\l}{L}{\l'}{-2}{0}{2}^2 + \Wjm{\l}{L}{\l'}{-2}{0}{2}\Wjm{\l}{L}{\l'}{2}{0}{-2} \bigg]
	\notag \\
	&= \INT{}{\mu}{}{-1}{1} \pi L(L+1)\sum_{\l\l'} \frac{2\l+1}{4\pi}A_\l\frac{2\l'+1}{4\pi}B_{\l'}
		4 (d^\l_{-2,-2}d^L_{00}d^{\l'}_{22} + d^\l_{-2,2}d^L_{00}d^{\l'}_{2,-2})
	\notag \\ 
	&= \INT{}{\mu}{}{-1}{1} 4\pi (\xi_{-2,-2}[A]\xi_{22}[B]+\xi_{-2,22}[A]\xi_{22}[B])d^L_{00} 
	\,, 
}
%----------------------------------------------------------------------------------------------------%
and
%----------------------------------------------------------------------------------------------------%
\al{
	\Gamma_L^{(-),\alpha}[A,B] 
	&= \pi \sum_{\l\l'}\frac{2\l+1}{4\pi}A_\l\frac{2\l'+1}{4\pi}B_{\l'}
		8[1+(-1)^{\l+L+\l'}] \Wjm{\l}{L}{\l'}{-2}{0}{2} \Wjm{\l'}{L}{\l}{-2}{0}{2}
	\notag \\
	&= \pi \sum_{\l\l'}\frac{2\l+1}{4\pi}A_\l\frac{2\l'+1}{4\pi}B_{\l'} 
	\notag \\
	&\quad\times 8\Wjm{\l}{L}{\l'}{-2}{0}{2} \bigg[\Wjm{\l}{L}{\l'}{-2}{0}{2}+\Wjm{\l}{L}{\l'}{2}{0}{-2}\bigg]
	\notag \\
	&= \INT{}{\mu}{}{-1}{1} \pi \sum_{\l\l'}\frac{2\l+1}{4\pi}A_\l\frac{2\l'+1}{4\pi}B_{\l'} 
		4[d^\l_{-2,-2}d^L_{00}d^{\l'}_{22} + d^\l_{-2,2}d^L_{00}d^{\l'}_{2,-2}]
	\notag \\ 
	&= \INT{}{\mu}{}{-1}{1} 4\pi (\xi_{-2,-2}[A]\xi_{22}[B]+\xi_{-2,2}[A]\xi_{2,-2}[B]) d^L_{00}
	\,. 
}
%----------------------------------------------------------------------------------------------------%

%::::::::::::::::::::::::::::::::::::::::::::::::::::::::::::::::::::::::::::::::::::::::::::::::::::%
\subsection{Kernel Functions: Tau}
%::::::::::::::::::::::::::::::::::::::::::::::::::::::::::::::::::::::::::::::::::::::::::::::::::::%


%----------------------------------------------------------------------------------------------------%
\al{
	\Sigma_L^{(0),\tau}[A,B] &= \pi \sum_{\l\l'} \frac{2\l+1}{4\pi}A_\l \frac{2\l'+1}{4\pi}B_{\l'} 
		2 (1+(-1)^{\l+L+\l'})\Wjm{\l}{L}{\l'}{0}{0}{0}^2
	\notag \\
		&= \INT{}{\mu}{}{-1}{1} \pi \sum_{\l\l'}  \frac{2\l+1}{4\pi}A_\l \frac{2\l'+1}{4\pi}B_{\l'}
		2 d^\l_{00}d^L_{00}d^{\l'}_{00}
	\notag \\
		&= \INT{}{\mu}{}{-1}{1} 2\pi \zeta_{00}[A]\zeta_{00}[B] d^L_{00}
	\,.
}
%----------------------------------------------------------------------------------------------------%

%----------------------------------------------------------------------------------------------------%
\al{
	\Gamma_L^{(0),\tau}[A,B] &= 4\pi\sum_{\l\l'} \frac{2\l+1}{4\pi}A_\l \frac{2\l'+1}{4\pi}B_{\l'} 
		\frac{1+(-1)^{\l+L+\l'}}{2}\Wjm{\l}{L}{\l'}{0}{0}{0}\Wjm{\l}{L}{\l'}{0}{0}{0}
	\notag \\
		&= \Sigma_L^{(0),\tau}[A,B]
	\,. 
}
%----------------------------------------------------------------------------------------------------%

% References %
\bibliographystyle{mybst}
\bibliography{main}

\end{document}

